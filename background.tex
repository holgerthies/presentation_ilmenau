\section{Background}
\subsection{Computing on reals}

\begin{frame}
	% In real complexity theory real numbers are modeled by machines approximating them.\\
	% \pause
	% Call a function $d:\NN\to\ZZ$ an approximation function for $x$ if
	% \[ \left|\frac{d(n)}{2^n}-x\right| \leq 2^{-n}. \]
	% \pause
	% we will often write $d_n$ for $d(n)$.
	% \pause
	\begin{definition}
		A real number $x$ is called computable if there is a machine that computes approximations.
	\end{definition}
% 	\pause
% 	It is called polynomial time computable, if there is a machine computing the values $d_n$ in time polynomial in $n$.
% 	\pause
% 	(This means the input $n$ is considered to be in unary).
% \end{frame}

%\begin{frame}
\begin{minipage}{.45\textwidth}
		\begin{figure}
		\centering
		\begin{tikzpicture}<2->
				\path (0,0) rectangle (3.5,-2.7);
			%x
				\draw<2->[thick] (0,0) rectangle (3,-2);
				\node<2-3> at (1.5,-1) {$x$};
				\draw<2-> (.5,-2) -- (.5,-2.2);
				\draw<2->[-triangle 90] (.5,-2.5) -- (.5,-2.2);
				\node<2-> at (.25,-2.25) {$1^n$};
				\draw<2->[-triangle 90] (2.5,-2) -- (2.5,-2.3);
				\draw<2-> (2.5,-2.3) -- (2.5,-2.5);
				\node<2-3> at (2.75,-2.25) {$d_n$};
				%\node<4-> at (2.85,-2.25) {$d_{n,i}$};
				%\node<4-> at (1.5,-1) {$(a_i)$};
				%\draw<4-> (1,-2) -- (1,-2.2);
				%\draw<4->[-triangle 90] (1,-2.5) -- (1,-2.2);
				%\node<4-> at (1.25,-2.25) {$1^i$};
		\end{tikzpicture}
		\end{figure}
		\begin{itemize}
			\item<3-> polynomial time computability
		\end{itemize}
	\end{minipage}
	\hfill
	\begin{minipage}{.45\textwidth}
		\onslide<2->{$d_n$ rational approximations}
		\only<2->{\[ \left|d_{n} - x\right|\leq 2^{-n}. \]}
		% \only<5->{\[ \left|d_{n,i} - a_i\right|\leq 2^{-n}. \]}
	\end{minipage}
\end{frame}
\subsection{Real functions}

\begin{frame}
	\begin{minipage}{.45\textwidth}
		\begin{tikzpicture}
			\path (-1.7,1.7) rectangle (3.7,-5.2);
			%x
				\draw<1-> (0,0) rectangle (3,-2);
				\node<1-> at (1.5,-1) {$x$};
				\draw<1-> (1,-2) -- (1,-2.2);
				\draw<1->[-triangle 90] (1,-2.5) -- (1,-2.2);
				\node<1-> at (.5,-2.25) {$1^n$};
				\draw<1->[-triangle 90] (2,-2) -- (2,-2.3);
				\draw<1-> (2,-2.3) -- (2,-2.5);
				\node<1-> at (2.5,-2.25) {$d_n$};
			%f
				\draw<1->[thick] (0,-2.5) rectangle (3,-4.5);
				\node<1-> at (1.5,-3.5) {$f$};
				\draw<1-> (1,-4.5) -- (1,-4.7);
				\draw<1->[-triangle 90] (1,-5) -- (1,-4.7);
				\node<1-> at (.5,-4.75) {$1^m$};
				\draw<1->[-triangle 90] (2,-4.5) -- (2,-4.8);
				\draw<1-> (2,-4.8) -- (2,-5);
				\node<1-> at (2.5,-4.75) {$d'_m$};
			%f(x)
				\draw<1-> (-1.5,.5) rectangle (3.5,-4.5);
				\node<1-> at (-.75,-2) {$f(x)$};
		\end{tikzpicture}
	\end{minipage}
	\hfill
	\begin{minipage}{.47\textwidth}
		\begin{definition}
			\small
			A function $f:[0,1]\to\RR$ is called computable, \newline if there is a machine that,\newline when given oracle access to approximations to the argument $x$,\newline returns approximations to the value $f(x)$.
		\end{definition}
		\begin{itemize}
			\item<2-> Computable functions are continuous.
		\end{itemize}
	\end{minipage}
\end{frame}
\subsection{iRRAM}
\begin{frame}[<+->]
\frametitle{iRRAM}
\begin{itemize}[<+->]
\item iRRAM is a C++ framework for exact real computations
\item Ordinary C++ is extended by datatype REAL for computing with real numbers
\item Usual arithmetic operations and other functions like abs, power, root, modulo, exp, log, sin, cos are already implemented
\end{itemize}
\end{frame}
\begin{frame}[<+->][fragile]
\frametitle{Example: \irram}
\begin{example}
\begin{lstlisting}
REAL series(int n){
  return power(REAL(2), -n);
}
REAL xinv_approx(long p, REAL& x){
  int N=-2*p+3;
  REAL ans = 0.0;
  for(int i=0; i<=N; i++)
    ans += series(i)*power(x,i);
  return ans;
}
REAL xinv(REAL& x) { return limit(xinv_approx, x);}
\end{lstlisting}
\end{example}
\end{frame}
