
\subsection{Implementation}
%\begin{frame}
%\frametitle{Classes}
%\code{POLY<coeff\_type>}
%\begin{itemize}
%\item<1-> A class for polynomials with coefficients of given type
%\item<2-> List of coefficients is internally represented as a vector
%\end{itemize}
% \onslide<3->{
%\code{POWERSERIES<coeff\_type>}
%}
%\begin{itemize}
%\item<3-> A class for powerseries
%\item<4-> Represented as a pointer to a sequence, i.e. a function from \code{INTEGER} to \code{coeff\_type}
%\end{itemize}
%\end{frame}
\begin{frame}[<+->]
\frametitle{Classes}
\code{BA\_ANA<coeff\_type>}
\begin{itemize}
\item A user-friendly class for analytic functions on a closed disc of rational radius
\item Standard operators \code{+}, \code{-}, \code{*} (both scalar multiplication and multiplication) are overloaded
\item Integration, Differentiation and Evaluation are implemented
\end{itemize}
\pause
\code{ANALYTIC\_RECT}
\begin{itemize}
\item A class for complex functions analytic on $[-1,1]$
\item Integration, Differentiation, Evaluation, \code{+},\code{-},  \code{*} also implemented  
\item Uses analytic continuation
\end{itemize}
\end{frame}
\begin{frame}[<+->]
\frametitle{Running Time Evaluation}
\begin{itemize}
	\item For both classes the running time analysis for different parameters and input functions were performed
	\item The running times were in accordance to what was expected from theoretical analysis
	\item The running time of iterated analytic continuation grows exponentially in the number of iterations 
	\item Evaluating a function using analytic continuation is therefore very slow
	\item It is not really usable (yet)
\end{itemize}
\end{frame}


