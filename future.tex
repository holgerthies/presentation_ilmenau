\section{Future Work}
\subsection{Overview}
\begin{frame}[<+->]
\frametitle{Goal}
The goal is to improve practical usability of exact real arithmetic by developing algorithms that are
\pause
\begin{itemize}
	\item Comparable in speed with their floating point variants
	\item Fully specified
	\item Provably correct
	\item closed under composition
\end{itemize}
\pause
For this, the behaviour (correctness, convergence, runtime) of advanced numerical algorithms has to be analyzed in detail. 
\end{frame}
\begin{frame}[<+->]
\frametitle{Ordinary Differential Equations}
An initial value problem is a differential equation combined with an initial 
condition.
$$
y'(t)=f(t,y(t)) \quad;\quad y(y_0)=t_0.
$$
\pause
Sophisticated floating point recipes exist to solve such initial value problems, but it is in general seen as a 
hard problem to check the validity of the obtained result.
\end{frame}
\begin{frame}
\frametitle{Ordinary Differential Equations}
\begin{theorem}[Kawamura, 2010]
Consider the IVP
$$
h'(t)=g(t,h(t)) \quad;\quad h(0)=0.
$$
\pause
There exist functions $g: [0,1] \times [-1,1] \to \RR$ and $h: [0,1] \to [-1,1]$
such that $g$ is computable in polynomial time and Lipschitz continuous
but $h$ is PSPACE-hard.
\end{theorem}
\pause
$h$ can be solved in PSPACE by the Euler method (Ko, 1983).\\
There is a gap between theory and practice!
\end{frame}
\begin{frame}[<+->]
\frametitle{Closing the gap}
\begin{itemize}
%	\item empirical evidence suggests that typical ODEs are apparently solved more efficient by sophisticated approaches
	\item Parametrized analysis of such approaches will be performed, taking
into account the number of steps and the intermediate precision required to attain guaranteed
output error $2^{-n}$ in worst-case.
	\item Methods will be implemented and empirically evaluated
	\item A first step: ODEs with analytic right hand side
\end{itemize}
\end{frame}